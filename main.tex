% Queremos un artículo.

% -------------------------------------------------------------------
% Comienzo del preámbulo e inclusión de paquetes.
% El tipo de documento es "libro".
% El tamaño de letra se puede cambiar a 11pt.
% -------------------------------------------------------------------

\documentclass[a4paper,12pt,twoside]{book}

% -------------------------------------------------------------------
% Información del TFG: título, alumno, tutor.
% -------------------------------------------------------------------

\newcommand{\titulo}{Escritura del TFG:\\[0.2cm] una plantilla \LaTeX}
\newcommand{\tituloPlano}{Escritura del TFG: una plantilla \LaTeX}

\newcommand{\titulacion}{Grado en Ingeniería Informática}
\newcommand{\autor}{Nombre del autor}
\newcommand{\director}{Nombre del tutor}

\title{\titulo}
\author{\autor}

% -------------------------------------------------------------------
% Paquetes.
% -------------------------------------------------------------------

% Carga de paquetes.
% Para poder escribir acentos.
\usepackage{fontspec}
% Sistema babel para escritura en castellano, catalán e inglés.
% La opción es-tabla sustituye el rótulo Cuadro por Tabla.
\usepackage[english,catalan,spanish,es-tabla]{babel}
% Para que el índice tenga links.
\usepackage{hyperref}
% Para poder poner URLs y links.
\usepackage{url}
% Para manejar más cómodamente la geometría de la página.
\usepackage[a4paper]{geometry}
% Tipo de papel y márgenes (tamaño A4).
\usepackage{anysize}
\papersize{29.7cm}{21cm}
\marginsize{3cm}{3cm}{2.35cm}{2.35cm}
% Para colocar imágenes flotantes.
\usepackage{wrapfig}
% Para poder tener subfiguras y para tener listings con caption bonito.
\usepackage{graphicx}
\usepackage{caption}
\usepackage{subcaption}
% Para tener cabeceras y pies de página personalizados.
\usepackage{fancyhdr}
% Para dibujar grafos.
\usepackage{tikz}
% Librerías de Tikz para trabajar más cómodamente con grafos.
\usetikzlibrary{arrows,shapes,snakes,automata,backgrounds,petri,positioning,calc,fit}
% Para poder dibujar gráficas.
\usepackage{pgfplots}
% Para tener múltiples columnas.
\usepackage{multicol}
% Para poder usar mathbb, esto es, poner símbolos como el de números naturales.
\usepackage{amsfonts}
% Para poder usar matrices.
\usepackage{amsmath}
% Para usar separador decimal de coma.
\usepackage{icomma}
% Para poder insertar código.
\usepackage{listings}
% Para poner prefijos en las enumeraciones.
\usepackage{enumitem}
% Para poder usar una imagen como fondo de una página.
\usepackage[pages=some]{background}
% Para que las secciones estén alineadas a la derecha.
\usepackage[raggedright]{titlesec}
% Para tener tablas que ocupen múltiples páginas.
\usepackage{longtable}
% Para poder poner pseudocódigo.
\usepackage[spanish,onelanguage]{algorithm2e}
% Para poner el título de algoritmos correctamente.
\usepackage{xpatch}
% Para tener apéndices.
\usepackage[toc,page]{appendix}
% Para poder poner el símbolo del euro.
\usepackage{eurosym}
% Para poder tener verbatim con borde.
\usepackage{fancyvrb}
% Para dibujar árboles de gramáticas.
\usepackage{qtree}
% Para poner Lorem Ipsum.
\usepackage{lipsum}

\newenvironment{Figure}
  {\par\medskip\noindent\minipage{\linewidth}}
  {\endminipage\par\medskip}
% Solarized colors.
\input{headers/solarized}
% Para tener listings con caption bonito.
\DeclareCaptionFont{white}{\color{white}}
\DeclareCaptionFormat{listing}{
	\colorbox{sbase00}{
		\parbox{\textwidth}{#1#2#3}
	}
}
\captionsetup[lstlisting]{
	format=listing,
	labelfont=white,
	textfont=white,
	margin=0pt,
	skip=-0.35mm
}
% Para tener listings bonitos.
\lstset{
	basicstyle=\fontsize{9}{11}\color{sbase00}\ttfamily, % Tipografía
	%numbers=left,               % Posición del número de línea.
	numberstyle=\tiny,          % Tamaño del número de línea.
	%stepnumber=2,               % Saltos entre números de línea.
	numbersep=5pt,              % Tamaño del número de línea.
	tabsize=4,                  % Anchura de las tabulaciones.
	extendedchars=true,         %
	breaklines=false,           % Romper líneas automáticamente.
	% Colores.
	%backgroundcolor=\color{sbase3},
    keywordstyle=\color{sred},
    commentstyle=\color{sbase1},
    stringstyle=\color{sblue}\ttfamily,
    numberstyle=\color{sviolet},
    identifierstyle=\color{sbase00},
    %--------
	frame=b,                    % Borde inferior.
	showspaces=false,           % No resaltar espacios.
	showtabs=false,             % No resaltar tabs.
	xleftmargin=6.75mm,         % Ajusta el margen izquierdo.
	xrightmargin=-3.5mm,        % Ajusta el margen derecho.
	framexleftmargin=17pt,      % Ajusta el relleno izquierdo.
	framexrightmargin=5pt,      % Relleno derecho.
	framexbottommargin=4pt,     % Relleno inferior.
	showstringspaces=false      % Resaltar espacios.
}
% Para diagramas de caja.
\newlength\myframesep
\setlength\myframesep{8pt}

\pgfdeclarelayer{background}
\pgfsetlayers{background,main}

\pgfkeys{
    /tikz/node distance/.append code={
        \pgfkeyssetvalue{/tikz/node distance value}{#1}
    }
}

\newcommand\widernode[5][blueb]{
\node[
        #1,
        inner sep=0pt,
        shift=($(#2.south)-(#2.north)$),
        yshift=-\pgfkeysvalueof{/tikz/node distance value},
        fit={(#2) (#3)},
        label=center:{\sffamily\bfseries\color{white}#4}] (#5) {};
}
% Para tener el fondo preparado para la portada.
\backgroundsetup{
scale=1,
color=black,
opacity=1,
angle=0,
contents={%
	\includegraphics[width=\paperwidth,height=\paperheight]{frontpage/bg}
}%
}

% Para evitar líneas huérfanas y viudas.
\clubpenalty=10000
\widowpenalty=10000

% Operadores matemáticos útiles.
\DeclareMathOperator*{\argmax}{arg\,max}
\DeclareMathOperator*{\argmin}{arg\,min}

% Para mostrar el año pasado.
\newcommand\LastYear{\advance\year by -1 \the\year\advance\year by 1}

% Separación entre párrafos.
\setlength{\parskip}{1em}

% -------------------------------------------------------------------
% Estilo personalizado para las cabeceras y pies de página.
% -------------------------------------------------------------------
\pagestyle{fancy}

\fancyhf{}
\renewcommand{\headrulewidth}{0pt}

\fancyhead[RO,LE]{\small\tituloPlano}
\fancyfoot[LE]{
	\raisebox{-10mm}{
		\includegraphics[height=0.64cm]{headers/icon-etsinf}
		\hspace{0.2cm}
		\raisebox{1.75mm}{\small\textbf{\thepage}}
	}
}
\fancyfoot[RO]{
	\raisebox{-10mm}{
		\raisebox{1.75mm}{\small\textbf{\thepage}}
		\hspace{0.2cm}
		\includegraphics[height=0.64cm]{headers/icon-etsinf}
	}
}

% -------------------------------------------------------------------
% Glosario.
% -------------------------------------------------------------------
\usepackage[toc]{glossaries}
\makeglossaries
\usepackage[xindy]{imakeidx}
\makeindex
\newacronym{upv}{UPV}{Universidad Politécnica de Valencia}
\newacronym{tfg}{TFG}{Trabajo de Fin de Grado}

\newglossaryentry{latex}
{
    name={\LaTeX},
    description={Sistema de composición de textos, orientado a la creación de documentos escritos que presenten una alta calidad tipográfica}
}


% -------------------------------------------------------------------
% Documento.
% -------------------------------------------------------------------
\begin{document}
% -------------------------------------------------------------------

% -------------------------------------------------------------------
% Portada.
% -------------------------------------------------------------------
% -------------------------------------------------------------------
% Portada.
% -------------------------------------------------------------------
\begin{titlepage}
	\begin{center}

		\pagestyle{empty}
		\BgThispage
% -------------------------------------------------------------------

		% -----------------------------------------------------------
		% Logos UPV y ETSINF.
		% -----------------------------------------------------------

		\begin{minipage}{0.49\linewidth}
			\begin{flushleft}
				\includegraphics[height=1.5cm]{frontpage/logo-upv}
			\end{flushleft}
		\end{minipage}
		\begin{minipage}{0.49\linewidth}
			\begin{flushright}
				\includegraphics[height=1.5cm]{frontpage/logo-etsinf}
			\end{flushright}
		\end{minipage}
		
		\vspace{2cm}
		
		% -----------------------------------------------------------
		% Nombre de la escuela.
		% -----------------------------------------------------------
		
		\begin{color}{schoolName}
		\large
		Escola Tècnica Superior d'Enginyeria Informàtica\\[0.2cm]
		Universitat Politècnica de València\\[4cm]
		\end{color}
		
		% -----------------------------------------------------------
		% Título del proyecto y titulación.
		% -----------------------------------------------------------
		
		{\LARGE \bfseries \titulo}\\[1.5cm]
		{\large Trabajo Fin de Grado}\\[0.4cm]
		\textbf{\textcolor{schoolName}{\large\titulacion}}\\[5.0cm]
		
		% -----------------------------------------------------------
		% Autor, director y fecha.
		% -----------------------------------------------------------
		
		\begin{flushright} \large
			\textbf{Autor:} \autor\\[0.4cm]
			\textbf{Tutor:} \director\\[0.6cm]
			\LastYear-\the\year
		\end{flushright}		
		
% -------------------------------------------------------------------
		\clearpage
	\end{center}
\end{titlepage}
% -------------------------------------------------------------------

% -------------------------------------------------------------------
% Reverso de la portada.
% -------------------------------------------------------------------
\newpage
\thispagestyle{empty}
\phantom{Reverso de la portada}

% -------------------------------------------------------------------
% Agradecimientos.
% -------------------------------------------------------------------
%\include{frontpage/thanks}

% -------------------------------------------------------------------
% Reverso de los agradecimientos.
% -------------------------------------------------------------------
%\newpage
%\thispagestyle{empty}
%\phantom{Reverso de los agradecimientos}

% -------------------------------------------------------------------
% Resumen.
% -------------------------------------------------------------------
\newpage
\thispagestyle{empty}
% -------------------------------------------------------------------
% Resumen del TFG: castellano, valenciano e inglés.
% -------------------------------------------------------------------

\selectlanguage{spanish}

\begin{flushright}
	\noindent \textbf{\Huge{Resumen}}
	\vspace*{0.3cm}
	\hrule
	\vspace*{-0.3cm}
\end{flushright}

Plantilla \gls{latex} para el \acrfull{tfg} de la \acrshort{upv}.

\noindent \textbf{Palabras clave:} \LaTeX, \acrshort{tfg}, \acrshort{upv}.

\vspace{2cm}

\selectlanguage{catalan}

\begin{flushright}
	\noindent \textbf{\Huge{Resum}}
	\vspace*{0.3cm}
	\hrule
	\vspace*{-0.3cm}
\end{flushright}

Plantilla \gls{latex} per al \acrfull{tfg} de la \acrshort{upv}.

\noindent \textbf{Paraules clau:} \LaTeX, \acrshort{tfg}, \acrshort{upv}.

\vspace{2cm}

\selectlanguage{english}

\begin{flushright}
	\noindent \textbf{\Huge{Abstract}}
	\vspace*{0.3cm}
	\hrule
	\vspace*{-0.3cm}
\end{flushright}

\gls{latex} template for \acrshort{upv} degree's final project.

\noindent \textbf{Keywords:} \LaTeX, \acrshort{tfg}, \acrshort{upv}.

% -------------------------------------------------------------------
% Reverso del resumen (incluir solo si es necesario).
% -------------------------------------------------------------------
\newpage
\thispagestyle{empty}
\phantom{Reverso del resumen}

% -------------------------------------------------------------------
% Se escoge la lengua en que se escribe el documento.
% -------------------------------------------------------------------
\selectlanguage{spanish}

% -------------------------------------------------------------------
% Índice general.
% -------------------------------------------------------------------
\newpage
\pagenumbering{roman}
\renewcommand{\contentsname}{Índice general\\ \vspace*{0.3cm}\hrule\vspace*{-2cm}}
\setcounter{tocdepth}{1}
\tableofcontents

% -------------------------------------------------------------------
% Reverso del índice de contenidos (incluir solo si es necesario).
% -------------------------------------------------------------------
\newpage
\thispagestyle{empty}
\phantom{Reverso del índice de contenidos}

% -------------------------------------------------------------------
% Índice de figuras.
% -------------------------------------------------------------------
\newpage
\pagenumbering{roman}
\renewcommand{\listfigurename}{Índice de figuras\\ \vspace*{0.3cm}\hrule\vspace*{-0.3cm}}
\listoffigures

% -------------------------------------------------------------------
% Reverso de la lista de figuras (incluir solo si es necesario).
% -------------------------------------------------------------------
\newpage
\thispagestyle{empty}
\phantom{Reverso de la lista de figuras}

% -------------------------------------------------------------------
% Lista de tablas.
% -------------------------------------------------------------------
\newpage
\pagenumbering{roman}
\renewcommand{\listtablename}{Índice de tablas\\ \vspace*{0.3cm}\hrule\vspace*{-0.3cm}}
\listoftables

% -------------------------------------------------------------------
% Reverso de la lista de tablas (incluir solo si es necesario).
% -------------------------------------------------------------------
\newpage
\thispagestyle{empty}
\phantom{Reverso de la lista de tablas}

% -------------------------------------------------------------------
% Lista de algoritmos.
% -------------------------------------------------------------------
\newpage
\pagenumbering{roman}
\renewcommand*{\listalgorithmcfname}{Índice de algoritmos\\ \vspace*{0.3cm}\hrule\vspace*{-0.3cm}}
\listofalgorithms
\addtocontents{loa}{\def\string\figurename{Algorithm}}

% -------------------------------------------------------------------
% Reverso de la lista de algoritmos (incluir solo si es necesario).
% -------------------------------------------------------------------
\newpage
\thispagestyle{empty}
\phantom{Reverso de la lista de algoritmos}

% -------------------------------------------------------------------
% Definición de la parte principal del documento.
% -------------------------------------------------------------------
\mainmatter
\makeatletter
\renewcommand{\ps@plain}{\ps@fancy}
\def\@makechapterhead#1{%
	\vspace*{50\p@}%
		{\parindent \z@ \raggedright \normalfont
		\interlinepenalty\@M
		\begin{flushright}
    		\Huge \bfseries \thechapter.~#1\\ \vspace*{0.6cm}\hrule\vspace*{-0.6cm}
		\end{flushright}
		\vskip 40\p@
	}
}
\makeatother
\pagestyle{fancy}

% -------------------------------------------------------------------
% A partir de aquí comienzan los capítulos.
% -------------------------------------------------------------------

% Introducción.
\chapter{Introducción}

Lorem ipsum dolor sit amet, consectetur adipiscing elit. Donec eget libero a urna pellentesque hendrerit a nec ante. Aliquam id mi id sapien molestie iaculis et eu diam. Vivamus purus ipsum, blandit lobortis blandit non, tempus nec ligula. Etiam sollicitudin eleifend velit et varius. Interdum et malesuada fames ac ante ipsum primis in faucibus. Pellentesque mollis magna dolor, a convallis nisi egestas eu. Duis ut ex sit amet elit lacinia facilisis. Aenean ligula orci, pellentesque eu gravida in, gravida et elit. Proin laoreet at neque nec imperdiet. Nulla lacinia est quis urna tincidunt laoreet. Cum sociis natoque penatibus et magnis dis parturient montes, nascetur ridiculus mus. Etiam mi nisl, tempus eget velit sed, pellentesque porttitor enim. Phasellus vestibulum ligula dapibus, auctor tellus vitae, dignissim metus. Aliquam congue ornare nunc et scelerisque.

\section{Motivación}

Quisque tempus arcu ut quam pharetra, in accumsan risus pharetra. Integer eleifend egestas lorem in rutrum. Nunc lacinia sagittis nunc a ullamcorper. Donec pretium magna eget imperdiet accumsan. Sed tincidunt ex sed dignissim commodo. Aliquam eu varius urna. Curabitur eget tempor nibh. Curabitur ligula magna, condimentum quis urna at, consectetur interdum nibh. Pellentesque consectetur nisl et arcu euismod, et consectetur magna imperdiet. Etiam malesuada risus a porttitor semper. Vestibulum fermentum quis dui vel cursus. Fusce sed ipsum porttitor orci iaculis porttitor.

\section{Objetivos}

Morbi luctus lectus non elementum tincidunt. Donec sit amet arcu vel tellus dapibus cursus. Phasellus suscipit enim eget neque blandit, a tempus urna finibus. Etiam nec porta magna, a interdum elit. Pellentesque in gravida eros, in sollicitudin nulla. Mauris aliquet sollicitudin vehicula. Aliquam et dignissim elit. Vivamus accumsan vehicula massa non suscipit. Mauris pharetra arcu eget nibh efficitur, non tincidunt nunc maximus. Praesent id ultrices tortor, a efficitur est. Sed at nisi sit amet nisi sollicitudin euismod eu a orci. Vivamus elit nisi, aliquet nec venenatis sit amet, tincidunt laoreet est. Curabitur ornare risus quis luctus sodales. Nam aliquet cursus libero, non iaculis urna accumsan a.

Integer et lorem faucibus, interdum mi commodo, interdum est. Ut et gravida urna. Nam aliquam mollis lobortis. Sed ultrices interdum aliquet. Suspendisse commodo nisi elit, ut interdum turpis varius nec. Nullam tincidunt nisi ac rutrum euismod. Duis semper ultricies neque, eget condimentum turpis vehicula sed. Ut quis sem aliquam, vulputate nulla ac, cursus nisl. Suspendisse sit amet elementum lorem. Proin eu bibendum arcu, quis facilisis ex. Ut volutpat, diam non convallis maximus, velit velit consequat justo, eu tincidunt sapien magna quis libero. Etiam ullamcorper eu leo sed sagittis. Aenean sit amet sagittis ex. Nullam bibendum orci ac lorem tristique, vitae malesuada ipsum molestie. In imperdiet nec erat sed scelerisque. Vestibulum finibus ornare eros vel laoreet.

\section{Organización del trabajo}

Nunc vehicula euismod eros. Suspendisse potenti. Pellentesque est neque, tincidunt ut imperdiet id, placerat vel nibh. Class aptent taciti sociosqu ad litora torquent per conubia nostra, per inceptos himenaeos. Fusce convallis massa eget risus laoreet, sit amet imperdiet lorem rhoncus. Donec non lacus ut dolor facilisis porta vitae ut risus. Duis suscipit sollicitudin convallis. Nam a risus eu ante pharetra vehicula. Donec non massa id nisl viverra sodales. Vestibulum accumsan augue ut rutrum faucibus. Nunc maximus ut ipsum a maximus. Pellentesque facilisis nisl eu orci pellentesque facilisis. Donec eget urna neque. Morbi ornare gravida lectus.



% -------------------------------------------------------------------
% Mostrar glosario.
% -------------------------------------------------------------------
\appendix
\newpage
\pagenumbering{roman}
\glsaddall
\printindex
\printglossary[toctitle=Glosario,title=Glosario\\ \vspace*{0.3cm}\hrule\vspace*{-2.3cm}]

% -------------------------------------------------------------------
% Mostrar bibliografía.
% -------------------------------------------------------------------
\appendix
\newpage
% Bibliografía.
\nocite{*}
\renewcommand{\bibname}{Bibliografía\\ \vspace*{0.3cm}\hrule\vspace*{-2.3cm}}

\addcontentsline{toc}{chapter}{Bibliografía}


\bibliographystyle{acm}
\bibliography{main}{}


% -------------------------------------------------------------------
% Mostrar anexos.
% -------------------------------------------------------------------

\newpage
\chapter{Anexo}
\label{sec:stopwords}

Esta capítulo se muestra como un anexo.

% -------------------------------------------------------------------
\end{document}