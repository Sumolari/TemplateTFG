% Para poder escribir acentos.
\usepackage{fontspec}
% Sistema babel para escritura en castellano, catalán e inglés.
% La opción es-tabla sustituye el rótulo Cuadro por Tabla.
\usepackage[english,catalan,spanish,es-tabla]{babel}
% Para que el índice tenga links.
\usepackage{hyperref}
% Para poder poner URLs y links.
\usepackage{url}
% Para manejar más cómodamente la geometría de la página.
\usepackage[a4paper]{geometry}
% Tipo de papel y márgenes (tamaño A4).
\usepackage{anysize}
\papersize{29.7cm}{21cm}
\marginsize{3cm}{3cm}{2.35cm}{2.35cm}
% Para colocar imágenes flotantes.
\usepackage{wrapfig}
% Para poder tener subfiguras y para tener listings con caption bonito.
\usepackage{graphicx}
\usepackage{caption}
\usepackage{subcaption}
% Para tener cabeceras y pies de página personalizados.
\usepackage{fancyhdr}
% Para dibujar grafos.
\usepackage{tikz}
% Librerías de Tikz para trabajar más cómodamente con grafos.
\usetikzlibrary{arrows,shapes,snakes,automata,backgrounds,petri,positioning,calc,fit}
% Para poder dibujar gráficas.
\usepackage{pgfplots}
% Para tener múltiples columnas.
\usepackage{multicol}
% Para poder usar mathbb, esto es, poner símbolos como el de números naturales.
\usepackage{amsfonts}
% Para poder usar matrices.
\usepackage{amsmath}
% Para usar separador decimal de coma.
\usepackage{icomma}
% Para poder insertar código.
\usepackage{listings}
% Para poner prefijos en las enumeraciones.
\usepackage{enumitem}
% Para poder usar una imagen como fondo de una página.
\usepackage[pages=some]{background}
% Para que las secciones estén alineadas a la derecha.
\usepackage[raggedright]{titlesec}
% Para tener tablas que ocupen múltiples páginas.
\usepackage{longtable}
% Para poder poner pseudocódigo.
\usepackage[spanish,onelanguage]{algorithm2e}
% Para poner el título de algoritmos correctamente.
\usepackage{xpatch}
% Para tener apéndices.
\usepackage[toc,page]{appendix}
% Para poder poner el símbolo del euro.
\usepackage{eurosym}
% Para poder tener verbatim con borde.
\usepackage{fancyvrb}
% Para dibujar árboles de gramáticas.
\usepackage{qtree}
% Para poner Lorem Ipsum.
\usepackage{lipsum}

\newenvironment{Figure}
  {\par\medskip\noindent\minipage{\linewidth}}
  {\endminipage\par\medskip}